\section{Related Work}

Several recent studies have contributed techniques to combine
system-wide HPC instrumentation data for integrated
analysis~\cite{Lockwood2017,Vazhkudai2017guide,Agelastos2014ldms,Kunkel2014siox,RIOT_2013}
. Vazhkudai et al. notably used a Splunk data warehouse to analyze system
logs and perform operational analytics on a large-scale storage system.
The cloud computing community has also identified the need to to unify
analysis capabilities across diverse environments~\cite{Vazhkudai2017guide}. The Dapper system
developed by Sigelman et al. combines comprehensive tracing with a novel
sampling method to observe critical paths in large cloud environments in great
detail~\cite{Sigelman2010dapper}.

Di et al.~\cite{7973730} and Park et al.~\cite{Park2017BigDM} have
investigated how to perform correlation analysis on HPC log data once it has been
collected.  Their analyses included compute and network resources, with a
particular emphasis on event logs. Inacio et al. analyzed performance variability using statistical analysis of file system read/write operations to conclude experimental environment and Lustre stripe settings impact performance applications. 
Several researchers have documented 
I/O performance variability anecdotes on leadership-scale
systems~\cite{Lofstead2010,Yildiz2016,carns2011understanding} and proposed
methods to combat it.  

\TODO{Adjust below to be more more clear about how we differe in scope, findings, and new analysis
techniques more so than methodology.}
Existing studies of application I/O and parallel file system performance and variability are based either on a small set of applications or on observations for a short duration. In this paper, we 
 build upon best-in-class previous efforts by combining system
monitoring, application monitoring, and explicit performance probing
into a portable, holistic framework without the need for additional database
infrastructure, and by introducing methods to help automate the task
of deriving actionable insight from these data sources. We have analyzed logs for an entire year on multiple parallel file systems, which is unprecedented in understanding HPC I/O performance variability.
