\section{Conclusion}

\TODO{Go for some really strong conclusions here.
Refer back to the abstract and introduction and either respond assertively to the questions/contributions we listed there, or prune them out.
We would like to have some outcome that is stronger than ``we analyzed some data and found X''}

\TODO{What are implications for state of practice?
Was particular instrumentation notably helpful or unhelpful?
What's the next step for what we've shown here?}

\TODO{Convert to prose:}

\begin{enumerate}
\item \textbf{Baseline performance and performance variation changes over time.}
We have shown that the baseline peak performance for a given I/O motif on a file systems can change over time as a result of factors such as system software updates and sustained workloads motivated by external factors such as end of allocation year.

\item \textbf{The magnitude and sign of correlation between performance and other measured metrics also varies over time.} 
We demonstrated that high CPU load can correlate with favorable performance under healthy file system conditions, and it can coincide with unfavorable performance when non-I/O workloads are impacting storage servers.

\item \textbf{Bandwidth, IOPS, and metadata contention are often confidently correlated with I/O performance problems are occurring.}
We were able to obtain statistically significant trends about isolated performance transients by aggregating simple binary classifications based on coincident observations of worst values within regions.
That said, some jobs defied classification using our binary classification method.
This indicates that we are still missing telemetry from important components of the I/O subsystem that contribute to performance variation.

\item The methods presented here are not dependent on SMAs, and alternative approaches for both partitioning time series data and classifying the measurements within regions can be replaced with more sophisticated methods.
What we have shown is that a great deal of information about I/O performance variation can be quantified using holistic I/O analysis and simple statistical methods.
\end{enumerate}