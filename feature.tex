\section{Data Summary \& Motivation}  \label{sec:features}

\begin{figure}
    \centering
    \includegraphics[width=0.9\columnwidth]{summary-boxplots}
    \vspace{-.15in}
    \caption{I/O performance grouped by test applications and read(R)/write(W) mode.  Whiskers represent the 5th and 95th percentiles.}
    \label{fig:summary-boxplots}
%   \vspace{-.3in}
\end{figure}

\begin{figure*}
    \centering
    \includegraphics[width=0.90\linewidth]{summary-heatmap}
    \vspace{-.2in}
    \caption{Performance of daily benchmarks normalized to each benchmark's peak observed performance on the specified storage system.  The y-axis labels show combinations of system, I/O motif, and mode (Read/Write).  Grey represents days on which no observations were made.  The two regions highlighted in green boxes are expanded upon in Figure \ref{fig:regions-heatmap}.}
    \label{fig:summary-heatmap}
\end{figure*}

\begin{figure}
    \centering
    \includegraphics[width=0.90\linewidth]{regions-heatmap}
    \vspace{-.2in}
    \caption{Examples of structure in the fraction of peak performance observations.  Color scale is the same as that in Figure \ref{fig:summary-heatmap}.  In (a), the vertical band on Sept 26 corresponds to a transient system-wide degradation on \mira.  The horizontal bands for the IOR file-per-process write workload (IOR/fpp(W)) and HACC write workload (HACC(W)) in (b) show a sustained performance problem for file-per-process write workloads on \cori.}
    \label{fig:regions-heatmap}
\end{figure}




\subsection{Statistical overview} \label{sec:features/summary}

% \TODO{Things that are important here: behavior over time, grouping data sources in a way that helps understand scope, focusing on fundamental \emph{changes} in performance rather than steady state, everything in one view.}

I/O performance is known to vary as a function of (a) application I/O pattern, (b) whether I/Os read or write, and (c) the architecture of the system on which I/O is happening~\cite{Lockwood2017, Xie2012}.
To remove the effects of these factors and enable us to focus on performance \emph{variation}, we express the performance of each of the 11,986 observations in terms of its \emph{fraction of peak performance}.
This fraction of peak performance is defined as the absolute performance (in bytes/sec) of an observation divided by the maximum absolute performance observed across all jobs with a common (a), (b), and (c) above.

The distribution of the fraction of peak performance measurements for four of the five systems tested is shown in Fig. \ref{fig:summary-boxplots} and demonstrates that the performance of active I/O performance probes on production file systems is highly dynamic.
However, visualizing the raw data of two of these performance distributions (Fig. \ref{fig:summary-heatmap}) reveals that performance variation is not randomly distributed over the year, and time-independent distributions of performance variation do not fully capture the nature of performance variability in production storage systems.
Several archetypical forms of correlated performance degradation observed in Fig. \ref{fig:summary-heatmap} are highlighted in Fig. \ref{fig:summary-heatmap} and fall into three broad categories of variation:

\begin{enumerate}[leftmargin=*]
\item Dark vertical bands, exemplified in the \mira data in Fig. \ref{fig:regions-heatmap}, represent transient system-wide issues that resulted in a uniform loss of performance for all applications tested that day.
\item Dark horizontal bands, shown in the \cori data, indicate a long-term degradation in performance that disproportionately affects a specific I/O motif or reads or writes.
\item Isolated dark blocks represent individual application runs where performance was poor for a very short period of time within a day.
\end{enumerate}

% - Baseline performance and variability are not constant over time.  Predictive models must therefore adapt over time as well.
The preponderance of these time-dependent phenomena underscore the observation that \textbf{baseline I/O performance and variability are not constant over time}, and
what may qualify as abnormally poor performance during one period of time may be the baseline performance expectation during another.

This has implications for both HPC operators and users.
For system architects and operators, identifying and remedying the root causes of variation requires classifying an absolute performance measurement with respect to the performance region in which it was observed to determine if it was the result of a long-term divergence from baseline peak performance or if it was caused by a transient issue in the system.
% From Phil: In the prose we can site several predictive model papers (Bing Xie's papers, Sandeep's papers, and others) and point out that the implication is that these models aren't just "set it and forget it."  Not knocking them or even saying they can't do it, but just pointing out it's not covered yet.
For users and application developers, it follows that the accuracy of parameterized I/O performance models~\cite{Xie2012,Madireddy2017} will degrade unless they are reparameterized as the I/O subsystem they model evolves.
Both of these cases justify the need for a systematic approach for identifying different regions of I/O performance to differentiate long-term factors and phenomena from short-term transients.





\subsection{Time-dependent analysis} \label{sec:features/timedependent}

\begin{figure}[t]
    \centering
    \includegraphics[width=1.0\columnwidth]{longterm-cscratch-hacc}
    \vspace{-.35in}
    \caption{Performance evolution of HACC file-per-process workload on \cori.  Red line is the overall mean (298 GiB/sec write, 204 GiB/sec read) and blue bars are raw performance measurements.}
    \label{fig:timeseries-baseline}
    % source: sc18_segments.ipynb
\end{figure}

To address this need for systematic partitioning of performance regions, we apply Simple Moving Averages (SMAs) to the data collected from active I/O performance.
This approach is often used in financial market technical analysis to attenuate the day-to-day volatility in the price of various assets and help identify larger trends in the price movements of the underlying assets~\cite{james1968monthly,gunasekarage2001profitability}.
Given a time window of width $w$, the SMA for performance at time $t$ is the arithmetic mean of the fraction peak performance over ${-0.5w <= t < +0.5w}$.
When chosen to be sufficiently short (${w_{short} \sim O(\textup{days})}$), the resulting $\textup{SMA}_{short}$ provides a rapid visual means to identify performance degradation or recovery that lasts for $O(\textup{days})$.

An example of an SMA ($w_{short}$ = two weeks) applied to the performance data collected from HACC tests run on \cori is shown in Fig. \ref{fig:timeseries-baseline}.
When contrasted with a time-independent summary statistic such as the overall mean performance of the entire year, the SMA clearly identifies the long period of degraded HACC write performance on \cori that was qualitatively shown in the bottom half of Fig. \ref{fig:summary-heatmap}.
The points at which the SMA rise above or below the global mean performance also provide quantitative measurements of the region of time when an underlying issue manifested;
in Fig. \ref{fig:timeseries-baseline}, these \emph{crossover points} fall on March 24 and August 10.
Cross-referencing these dates with the service history of \cori retrospectively revealed that the beginning and end of this long-term region of divergent performance coincided with major system software upgrades that also happened on March 24 and August 10.

Curiously, the performance of HACC read workload (Figure \ref{fig:timeseries-baseline}b) was unaffected during this time, demonstrating that not all workloads are affected by long-term variation equally.
This asymmetry, in combination with the bounding dates of this divergent region, allowed us to trace this specific issue to unintentional behavior introduced (and later fixed) in the Lustre software running on \cori during the system upgrades.
Although this particular case of long-term performance divergence was caused by an unexpected bug in system software, the reality of most production storage systems is that they are regularly patched and upgraded.
At minimum, the security requirements of the centers which run them drive system updates, and as exemplified by the widely publicized Spectre and Meltdown patches, such updates can have non-trivial effects on certain types of I/O.
Thus, \textbf{performance-impacting bugs and patches represent a significant source of time-dependent, long-term performance variation that must be accounted for in both retrospective performance analyses and forward-looking performance modeling parameterization.}





\subsection{Partitioning divergent behavior} \label{sec:features/partitioning}

The above performance analysis, enabled by comparing a short-range SMA to a long-range summary statistic, can be generalized by superimposing a second SMA ($\textup{SMA}_{long}$) with a longer window (${w_{long} \sim O(\textup{weeks})}$) on top of $\textup{SMA}_{short}$ (which captures variations ${O(\textup{days})}$).
Doing so allows us to examine short-term performance variations (e.g., a period of sustained bandwidth contention) in the context of longer-term trends (e.g., in the presence of a file system software regression) and determine the causes of variation \emph{within} different regions of divergence such as the period described in Section \ref{sec:features/timedependent}.
The points at which $\textup{SMA}_{short}$ intersects $\textup{SMA}_{long}$, termed \emph{crossover points}, also conveniently establish the boundaries of regions where short-term performance has diverged from longer-term performance and anomalous performance is prevailing.
We therefore introduce the notion of \emph{divergence regions} which are the periods of time bounded by two crossover points and capture correlated performance.

For the remainder of this study, we apply the concept of \emph{divergence regions}, bounded by the crossovers between $\textup{SMA}_{short}$ and $\textup{SMA}_{long}$, to systematically identify and characterize periods in time where anomalous performance was observed by the active I/O probes running across all of the test systems.
We define $\textup{SMA}_{short}$ to have $w_{short} = 2 \textup{ weeks}$ as in \ref{sec:features/timedependent} and $\textup{SMA}_{long}$ to have $w_{long} = 7 \textup{ weeks}$.
$w$ was chosen to be a multiple of seven days to align with the U.S. work week and ensure that weekends and weekdays were equally represented both SMA calculations.
That said, our choice of $w_{long} = 7 \textup{ weeks}$ was somewhat arbitrary, and the analysis and conclusions presented below were insensitive to changes of $\pm 1 \textup{ week}$.