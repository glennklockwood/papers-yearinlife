\section{Introduction / Extended Abstract}

I/O performance variation has been studied extensively, and a variety of conditions have been identified as contributing factors to poor I/O performance.  Most studies have focused on enumerating the sources of I/O performance loss at a single point in time, assuming that performance loss is a transient effect due to contention from other jobs.  However, recent work~\cite{Lockwood2017} has shown that performance variation can occur over periods of days as a result of systematic, longer-term conditions of a storage system.

Such systematic, long-term I/O performance variation remains much less well understood despite its relevance to scientific campaigns that may experience uneven throughput or resource consumption over the course of months or years.  In this work, we differentiate \emph{long-term performance variation} from \emph{short-term performance variation} and characterize the factors that contribute to long-term performance variation on a diverse range of large-scale production parallel storage systems.

To accomplish this, we sampled the performance of a set of I/O benchmarks and applications on parallel file systems operated at NERSC and ALCF over the course of a year.  The resulting data captures performance variation over multiple time scales, ranging from days to weeks, and allows us insight into how often performance loss is observed over longer periods of time.  By integrating holistic I/O monitoring throughout the year-long experiment, we then quantitatively show that common sources of short-term variation, such as bandwidth or metadata contention, play less significant role in long-term performance variation.

Given this multiscale nature of I/O performance variation, we conclude that the probabilistic effects of transient interference \emph{and} the autocorrelative effects of long-term variation are required to capture the full picture of performance variation on parallel storage systems.