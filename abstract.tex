% As a general rule, do not put math, special symbols or citations
% in the abstract
\begin{abstract}

At present, I/O performance analysis requires different tools to characterize individual components of the I/O subsystem, and institutional I/O expertise is relied upon to translate these disparate data into an integrated view of application performance.  This process is labor-intensive and not sustainable as the storage hierarchy deepens and system complexity increases.  To address this growing disparity, we have developed the Total Knowledge of I/O (TOKIO) framework to combine the insights from existing component-level monitoring tools and provide a holistic view of performance across the entire I/O stack.

A reference implementation of TOKIO, pytokio, is presented here.  Using monitoring tools included with Cray XC and ClusterStor systems alongside commonly deployed community-supported tools, we demonstrate how pytokio provides a lightweight foundation for holistic I/O performance analyses on two Cray XC systems deployed at different HPC centers.  We present results from integrated analyses that allow users to quantify the degree of I/O contention that affected their jobs and probabilistically identify unhealthy storage devices that impacted their performance.  We also apply pytokio to inspect data motion through NERSC's DataWarp burst buffer and demonstrate, for the first time, a tool that elucidates sources of performance variation when staging data between a production burst buffer and Lustre file system.

\end{abstract}