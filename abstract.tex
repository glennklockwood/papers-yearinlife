% As a general rule, do not put math, special symbols or citations
% in the abstract
\begin{abstract}

I/O performance is critical to productivity for data-intensive
scientific computing, but it is notoriously difficult to understand
and diagnose. This problem is exacerbated by an ongoing trend towards
deeper I/O hierarchies and more complex I/O architectures. Capturing
and visualizing system-wide performance telemetry is a key step towards
solving this problem, but is only a partial solution because it continues to
hinge upon specialized human expertise to identify and interpret problems
and turn raw data into actionable intelligence.

In this work we explore how production I/O instrumentation can be turned
into actionable insights for large-scale facility management using
a year's worth of telemetry from two leading HPC facilities as a case
study.  In doing so, we answer the following questions: Does intuition
and methodology derived from individual application anecdotes match the
reality of long-term production behavior? How do we detect critical inflection
points in performance?  How do we assess the scope and contributing
factors for a given a performance inflection point?  Can underlying trends
be identified beyond the noise of day-to-day performance variability?

We find that \TODO{...}

\end{abstract}
